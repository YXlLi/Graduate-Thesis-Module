\linespread{1.6}%行间距设置
\numberwithin{equation}{section}%公式按照章节标号
\renewcommand\thefigure{\thesection.\arabic{figure}}%图片按章节标号
\makeatletter
\@addtoreset{figure}{section}
\renewcommand{\figurename}{图}               % 对图表中的Fig进行中文翻译变为图
\renewcommand{\contentsname}{目录}           % 对Contents进行汉化为目录
\renewcommand\listfigurename{插\ 图\ 目\ 录} % 对List of Figures进行汉化为插图目录
\renewcommand\listtablename{表\ 格\ 目\ 录}  % 对List of Tables进行汉化表格目录
\setcounter{tocdepth}{2}%目录章节深度设置
%\setlength{\parskip}{0.5\baselineskip}% 设置空行换行后,上下两段文字间距
%\titlespacing*{section}{0pt}{9pt}{0pt}% 设置标题与段落间距
%=================================边框样式设置===================================
\lstset{
    %行号
    numbers=left,
    %背景框
    framexleftmargin=10mm,
    frame=none,
    %背景色
    %backgroundcolor=\color[rgb]{1,1,0.76},
    backgroundcolor=\color[RGB]{245,245,244},
    %样式
    keywordstyle=\bf\color{blue},
    identifierstyle=\bf,
    numberstyle=\color[RGB]{0,192,192},
    commentstyle=\it\color[RGB]{0,96,96},
    stringstyle=\rmfamily\slshape\color[RGB]{128,0,0},
    %显示空格
    showstringspaces=false
 }
%=================================================================================
%===============================页眉页脚设置================================
\fancypagestyle{plain}{
\fancyhead[RE]{\leftmark} % 在偶数页的右侧显示章名

\fancyhead[LO]{\rightmark} % 在奇数页的左侧显示小节名
\fancyhead[LE,RO]{~\thepage~} % 在偶数页的左侧,奇数页的右侧显示页码
% 设置页脚:在每页的右下脚以斜体显示书名
\fancyfoot[RO,RE]{\it \text{右下角内容}} %右下角增加\text{}中的内容
\renewcommand{\headrulewidth}{0.7pt} % 页眉与正文之间的水平线粗细
\renewcommand{\footrulewidth}{0pt}
\renewcommand\headrule{\hrule width \hsize height 2pt \kern 2pt \hrule width \hsize height 0.4pt}%页眉上双线
}

